
\section{Discrete Compounding}
\label{append_A1}
To illustrate the use of the formulas presented in Chapter 2, we present some numerical examples based on the data provided in \cite{smith2011bond}. For simplicity, we will ignore the additional challenges posed by day count conventions and assume a 30/360 calendar.

Consider the following table for reference. 

\begin{table}[ht]
\begin{center}
\begin{tabular}{cccccc}

\toprule
\textsc{Forward} & \textsc{Forward} & \textsc{Spot} & \textsc{Spot} & \textsc{Discount} \\
\textsc{Period} & \textsc{Rates} & \textsc{Period} & \textsc{Rates} & \textsc{Factors}\\
\toprule

0 $\times$ 3 & 0.500\% & 0 $\times$ 3 & 0.5000\% & 0.998751561 \\
3 $\times$ 6 & 1.582\% & 0 $\times$ 6 & 1.0406\% & 0.994817233 \\
6 $\times$ 9 & 2.669\% & 0 $\times$ 9 & 1.5827\% & 0.988223069 \\
9 $\times$ 12 & 3.765\% & 0 $\times$ 12 & 2.1272\% & 0.979007831 \\
12 $\times$ 15 & 3.747\% & 0 $\times$ 15 & 2.4506\% & 0.969922559 \\
15 $\times$ 18 & 4.405\% & 0 $\times$ 18 & 2.7757\% & 0.959357293 \\
18 $\times$ 21 & 5.069\% & 0 $\times$ 21 & 3.1025\% & 0.947352265 \\
21 $\times$ 24 & 5.743\% & 0 $\times$ 24 & 3.4316\% & 0.933943555 \\
\toprule

\end{tabular}
\end{center}
\caption[Numerical Example--Discrete Compounding Formulas]{A table of forward rates, spot rates, and discount factors given on a quarterly compounded (30/360) basis.}
\label{tab:numerical_ex_1}
\end{table}

\newpage 

\subsection{Discrete spot rates from discrete forward rates}

Given a series of forward rates, we can compute the discrete spot rates.

\textbf{Example:}
\begin{align}
    \left( 1 + \frac{r_d(3M)]}{4}\right) \left( 1 + \frac{f_d(0,3M,6M)}{4} \right) &= \left( 1 + \frac{r_d(6M)}{4} \right)^2 \\[6pt]
    \left( 1 + \frac{0.00500}{4}\right) \left( 1 + \frac{0.01582}{4} \right) &= \left( 1 + \frac{r_d(6M)}{4} \right)^2 \\[6pt]
    r_d(6M) &= 0.010406 \\
    &= 1.0406\%.
\end{align}

\subsection{Discount factors from discrete spot rates}

We can compute discount factors given a set of discrete spot rates.

\textbf{Example:}
\begin{align}
    Z(0,15M) &= \frac{1}{\left( 1 + \frac{r_d(15M)}{4} \right)^5}  \\[6pt]
    Z(0,15M) &= \frac{1}{\left( 1 + \frac{0.024506}{4} \right)^5} \\[6pt]
    Z(0,15M) &= 0.969922559.
\end{align}

\subsection{Discrete forward rates from discount factors}

We can compute the discrete forward rates given a set of discount factors.

\textbf{Example:}
\begin{align}
    f_d(0,18M,21M) &= \frac{\left( \frac{Z(0,18M)}{Z(0,21M)} - 1 \right)}{21M - 18M} \\[6pt]
    f_d(0,18M,21M) &= \frac{\left( \frac{0.959357293}{0.947352265} - 1 \right)}{0.25} \\[6pt]
    f_d(0,18M,21M) &= 0.05069 \\
    &= 5.069\%.
\end{align}

\newpage

\section{Continuous Compounding}
\label{append_A2}

\subsection{Continuous spot rates}

Suppose we have the discrete spot rates given in Table \ref{tab:numerical_ex_1}. Then we can convert them to continuously compounded spot rates.

\textbf{Example:}
\begin{align}
    r_c(9M) &= \frac{1}{9M} \ln \left( 1 + (9M) r_d(9M)  \right) \\[6pt]
    r_c(9M) &= \frac{1}{0.75} \ln \left( 1 + (0.75)(0.015827) \right) \\[6pt]
    r_c(9M) &= 0.015796 \\
    &= 1.5796\%.
\end{align}

Suppose we have the following table, with all data now presented on a continuous basis.

\begin{table}[ht]
\begin{center}
\begin{tabular}{cccccc}

\toprule
\textsc{Forward} & \textsc{Forward} & \textsc{Spot} & \textsc{Spot} & \textsc{Discount} \\
\textsc{Period} & \textsc{Rates} & \textsc{Period} & \textsc{Rates} & \textsc{Factors}\\
\toprule

0 $\times$ 3 & 0.4997\% & 0 $\times$ 3 & 0.4997\% & 0.998751561 \\
3 $\times$ 6 & 1.5788\% & 0 $\times$ 6 & 1.0392\% & 0.994817233 \\
6 $\times$ 9 & 2.6602\% & 0 $\times$ 9 & 1.5796\% & 0.988223069 \\
9 $\times$ 12 & 3.7475\% & 0 $\times$ 12 & 2.1216\% & 0.979007831 \\
12 $\times$ 15 & 3.7294\% & 0 $\times$ 15 & 2.4431\% & 0.969922559 \\
15 $\times$ 18 & 4.3811\% & 0 $\times$ 18 & 2.7661\% & 0.959357293 \\
18 $\times$ 21 & 5.0370\% & 0 $\times$ 21 & 3.0905\% & 0.947352265 \\
21 $\times$ 24 & 5.7020\% & 0 $\times$ 24 & 3.4170\% & 0.933943555 \\
\toprule

\end{tabular}
\end{center}
\caption[Numerical Example--Continuous Compounding Formulas]{A table of spot rates, forward rates, and discount rates given on a continuously compounded basis. Derived from the data given in \cite{smith2011bond}.}
\label{tab:numerical_ex_2}
\end{table}

\subsection{Discount factors from continuous spot rates}

We can compute discount factors given a set of continuous spot rates.

\textbf{Example:}
\begin{align}
    Z(0,18M) &= \exp ( -r_c(18M) (18M) ) \\[6pt]
    Z(0,18M) &= e^{-(0.027761)(1.5)} \\[6pt]
    Z(0,18M) &= 0.959357293.
\end{align}

\subsection{Continuous forward rates from discount factors}

We can compute the continuous forward rates given a set of discount factors.

\textbf{Example:}
\begin{align}
    f_c(0,9M,12M) &= \frac{\ln \left( \frac{Z(0,9M)}{Z(0,12M)} \right)}{12M - 9M} \\[6pt]
    f_c(0,9M,12M) &= \frac{\left( \frac{0.988223069}{0.979007831} \right)}{0.25} \\[6pt]
    f_c(0,9M,12M) &= 0.037475 \\
    &= 3.7475\%.
\end{align}

\section{Computing a Swap Price}
\label{append_A3}
To illustrate the formulas used to compute the swap fixed rate, which is equivalent to the price of a swap, we present a numerical example based on that given in \cite{smith2011bond}.

\subsection{Using discount factors only}

Suppose we have the discretely compounded spot rates and discount factors given in table \ref{tab:numerical_ex_1}. Selectively repeated here for the comfort of the reader.

\begin{table}[ht]
\begin{center}
\begin{tabular}{ccc}

\toprule
\textsc{Spot} & \textsc{Spot} & \textsc{Discount} \\
\textsc{Period} & \textsc{Rates} & \textsc{Factors}\\
\toprule

0 $\times$ 3 & 0.5000\% & 0.998751561 \\
0 $\times$ 6 & 1.0406\% & 0.994817233 \\
0 $\times$ 9 & 1.5827\% & 0.988223069 \\
0 $\times$ 12 & 2.1272\% & 0.979007831 \\
0 $\times$ 15 & 2.4506\% & 0.969922559 \\
0 $\times$ 18 & 2.7757\% & 0.959357293 \\
0 $\times$ 21 & 3.1025\% & 0.947352265 \\
0 $\times$ 24 & 3.4316\% & 0.933943555 \\
\toprule

\end{tabular}
\end{center}
\caption[Numerical Example--Swap Pricing using Discount Factors]{The repeated table of spot rates and discount rates given on a quarterly compounded (30/360) basis.}
\label{tab:numerical_ex_3}
\end{table}

\textbf{Example:}
We want to compute the swap fixed rate, $r_s$, of a two year, quarterly paid plain vanilla fixed-for-floating swap. 

\begin{align}
    r_s &= \frac{1 - Z(0,24M)}{\tau \sum_{i = 3M}^{24M} Z(0, i)} \\[6pt]
    &= \frac{0.066056445}{1.942843842} \\
    &= 0.033999873 \\
    &= 3.40\%.
\end{align}

\subsection{Using forward rates and discount factors}

Suppose we have the discretely compounded forward rates and discount factors given in table \ref{tab:numerical_ex_1}. Selectively repeated here for the comfort of the reader.

\begin{table}[ht]
\begin{center}
\begin{tabular}{ccccc}

\toprule
\textsc{Forward} & \textsc{Forward} & \textsc{Spot} & \textsc{Discount} \\
\textsc{Period} & \textsc{Rates} & \textsc{Period} & \textsc{Factors}\\
\toprule

0 $\times$ 3 & 0.500\% & 0 $\times$ 3 & 0.998751561 \\
3 $\times$ 6 & 1.582\% & 0 $\times$ 6 & 0.994817233 \\
6 $\times$ 9 & 2.669\% & 0 $\times$ 9 & 0.988223069 \\
9 $\times$ 12 & 3.765\% & 0 $\times$ 12 & 0.979007831 \\
12 $\times$ 15 & 3.747\% & 0 $\times$ 15 & 0.969922559 \\
15 $\times$ 18 & 4.405\% & 0 $\times$ 18 & 0.959357293 \\
18 $\times$ 21 & 5.069\% & 0 $\times$ 21 & 0.947352265 \\
21 $\times$ 24 & 5.743\% & 0 $\times$ 24 & 0.933943555 \\
\toprule

\end{tabular}
\end{center}
\caption[Numerical Example--Swap Pricing using Forward Rates]{The repeated table of forward rates and discount rates given on a quarterly compounded (30/360) basis.}
\label{tab:numerical_ex_4}
\end{table}

\textbf{Example:}
Let's compute the same swap fixed rate (SFR) for a two year, quarterly paid plain vanilla fixed-for-floating swap.

\begin{align}
    r_s &= \frac{\sum_{i=3M}^{24M} Z(0,i) f_d(0,i-1,i)]}{\sum_{i = 3M}^{24M} Z(0, i)} \\[6pt]
    &= \frac{0.264227437}{7.771375367} \\
    &= 0.034000087 \\
    &= 3.40\%.
\end{align}